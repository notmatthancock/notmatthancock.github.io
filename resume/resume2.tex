\documentclass[a4paper,10pt]{report} 
\usepackage[margin=0.5in]{geometry}

% Aligned paragraph table columns
\usepackage{array}
\newcolumntype{L}[1]{>{\raggedright\let\newline\\\arraybackslash\hspace{0pt}}m{#1}}
\newcolumntype{C}[1]{>{\centering\let\newline\\\arraybackslash\hspace{0pt}}m{#1}}
\newcolumntype{R}[1]{>{\raggedleft\let\newline\\\arraybackslash\hspace{0pt}}m{#1}}
\newcolumntype{"}{@{\hskip\tabcolsep\color{linecolor}{\vrule width 1pt}\hskip\tabcolsep}}
\makeatother

\usepackage{color}

% Link setup
\usepackage{hyperref}
\definecolor{linkcolor}{rgb}{0.2, 0.3, 0.5}
\definecolor{linecolor}{rgb}{0.5, 0.5, 0.5}
\hypersetup{unicode=true,
            pdftitle={Matthew C. Hancock},
            pdfborder={0 0 0},
            breaklinks=true,
            colorlinks,
            urlcolor=linkcolor,
            linkcolor=linkcolor}
\urlstyle{same} % don't use monospace font for urls

% No section numbers
\setcounter{secnumdepth}{0} 
% No page numbers
\pagestyle{empty}

% Set font.
\usepackage{fontspec}
%\setmainfont{URW Bookman L}
\setmainfont{URW Palladio L}

\usepackage{titlesec}
\titleformat{\section}{\Large\bfseries\raggedright}{}{0em}{}[\color{linecolor}{\titlerule[1pt]}]
\titlespacing{\section}{0pt}{10pt}{20pt}

\begin{document}

\section{Matthew C. Hancock}

\begin{tabular}{rl}
Address: & 317 Mabry St. Apt 1021 
           Tallahassee, FL 32304 \\
Phone: & 810.656.0561 \\
Email: & \href{mailto:mhancock@math.fsu.edu}{\nolinkurl{mhancock@math.fsu.edu}} \\
Website: & \url{https://notmatthancock.github.io}\end{tabular}

\section{Education}

\begin{tabular}{R{1in}"L{4in}}
    \emph{Fall 2012-present} & \textbf{Ph.D. Candidate} in Applied and Computational Mathematics \\
    & \textbf{Florida State University} (Tallahasse, FL)
\\[4pt] & \textbf{Focus}:  Machine learning and image-processing methods for lung image processing and analysis\\[4pt] & \textbf{Relevant coursework}: {\small Numerical Methods (interpolation, integration, ODEs/PDEs, linear algebra, and optimization), Machine Learning, Probability theory and Statistical Inference}\end{tabular}

\vspace{.15in}
\begin{tabular}{R{1in}"L{4in}}
    \emph{Spring 2012} & \textbf{B.S.} in Applied Mathematics with Computer Science focus \\
    & \textbf{Ferris State University} (Big Rapids, MI)
\end{tabular}

\vspace{.15in}


\section{Work Experience}

\begin{tabular}{R{1.5in}"L{4in}}
    \emph{Fall 2012-present} & \textbf{Teaching Assistant} at Florida State University (Tallahassee, FL) \\
\\[4pt] & \textbf{Responsibilities / Accomplishments}:
        \begin{itemize}
        \setlength\itemsep{0pt}
        {\small
            \item Instructor, Multi-variable Calculus (Summer 2017)
            \item Distinguished Teaching Assistant Award (2017)
            \item Assistant, Foundations of Computational Math (graduate level course) (Fall 2016, Spring 2017).
            \item Instructor, C++ computing seminar (Fall 2016).
            \item Instructor, Single-variable Calculus (Spring 2016, Summer 2016).
            \item Instructor, Precalculus (Fall 2014, Spring 2015).
            \item Recitation instructor, Discrete Mathematics (Fall 2015).
            \item Assistant, various math courses (College algebra, Liberal Arts math, Trigonometry, Business calculus).
        }
        \end{itemize}
\end{tabular}

\vspace{.15in}
\begin{tabular}{R{1.5in}"L{4in}}
    \emph{Fall 2011-Fall 2012} & \textbf{Web developer} at Occupational Research and Assessment (Big Rapids, MI) \\
\\[4pt] & \textbf{Responsibilities / Accomplishments}:
        \begin{itemize}
        \setlength\itemsep{0pt}
        {\small
            \item Acquired basic web development skills.
            \item Created and designed web systems for a number of third-party organizations using Ruby.
        }
        \end{itemize}
\end{tabular}

\vspace{.15in}
\begin{tabular}{R{1.5in}"L{4in}}
    \emph{Fall 2009-Fall 2011} & \textbf{Programming Tutor} at Ferris State University (Big Rapids, MI) \\
\\[4pt] & \textbf{Responsibilities / Accomplishments}:
        \begin{itemize}
        \setlength\itemsep{0pt}
        {\small
            \item Tutor for undergraduate introductory programming course taught with Python programming language.
        }
        \end{itemize}
\end{tabular}

\vspace{.15in}
\begin{tabular}{R{1.5in}"L{4in}}
    \emph{Spring 2010-Fall 2010} & \textbf{Calculus Tutor} at Ferris State University (Big Rapids, MI) \\
\\[4pt] & \textbf{Responsibilities / Accomplishments}:
        \begin{itemize}
        \setlength\itemsep{0pt}
        {\small
            \item Tutor for undergraduate calculus courses (mostly single-variable calculus material).
        }
        \end{itemize}
\end{tabular}

\vspace{.15in}


\section{Computer skills}

{ \begin{itemize} \item Python (NumPy, SciPy, Scikit Learn, Scikit Image, Theano, Cython) \end{itemize}}{ \begin{itemize} \item C++ \end{itemize}}{ \begin{itemize} \item Fortran \end{itemize}}{ \begin{itemize} \item \LaTeX \end{itemize}}{ \begin{itemize} \item Web (HTML, CSS, JavaScript; lesser: Ruby, PHP) \end{itemize}}{ \begin{itemize} \item SQLite, MySQL \end{itemize}}

\section{Journal Publications}

\begin{itemize}
    \item Matthew C. Hancock, Jerry F. Magnan. \textbf{Lung nodule malignancy classification using only radiologist quantified image features as inputs to statistical learning algorithms: probing the Lung Image Database Consortium dataset with two statistical learning methods}. \textit{SPIE Journal of Medical Imaging}. Dec. 2016. \url{http://dx.doi.org/10.1117/1.JMI.3.4.044504}\end{itemize}


\section{Conference Proceedings}

\begin{itemize}
    \item Matthew C. Hancock, Jerry F. Magnan. \textbf{Predictive capabilities of statistical learning methods for lung nodule malignancy classification using diagnostic image features: an investigation using the Lung Image Database Consortium dataset}. \textit{SPIE Medical Imaging Symposium, Computer-Aided Diagnosis Conference (Orlando, FL)}. Feb. 2017. \url{http://dx.doi.org/10.1117/12.2254446}\end{itemize}


\section{Talks Given}

\begin{itemize}
    \item Matthew C. Hancock, Jerry F. Magnan. \textbf{Predictive capabilities of statistical learning methods for lung nodule malignancy classification using diagnostic image features: an investigation using the Lung Image Database Consortium dataset}. \textit{SPIE Medical Imaging Symposium, Computer-Aided Diagnosis Conference (Orlando, FL)}. Feb. 2017. (talk associated with corresponding conference proceeding).\end{itemize}
\begin{itemize}
    \item Matthew C. Hancock. \textbf{10 FREE Python Libraries that will TOTALLY SHOCK you}. \textit{FSU Math Department Graduate Student Seminar}. Spring 2017. (Presentation of various Python library for scientific computing. The title is a spoof on clickbait journalism.). \url{http://notmatthancock.github.io/research/talks/gss-python/}\end{itemize}
\begin{itemize}
    \item Matthew C. Hancock. \textbf{A survey a of PDE-based methods for image segmentation }. \textit{FSU Math Department Graduate Student Seminar}. Spring 2016. \url{http://notmatthancock.github.io/research/talks/gss-pdes/}\end{itemize}


\section{Posters Presented}

\begin{itemize}
    \item Matthew C. Hancock, Jerry F. Magnan. \textbf{Lung nodule malignancy classification using diagnostic image features}. \textit{SIAM SEAS Conference (Tallahassee, FL)}. Spring 2017. \url{http://notmatthancock.github.io/research/pdf/siam-seas-2017.pdf}\end{itemize}

\vspace{0.25in}
 
\noindent\textit{\bf References available upon request.}

\end{document}

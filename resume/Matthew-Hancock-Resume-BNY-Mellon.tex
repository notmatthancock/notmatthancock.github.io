\documentclass[a4paper,10pt]{report} 
\usepackage[margin=1in]{geometry}

% Aligned paragraph table columns
\usepackage{array}
\newcolumntype{L}[1]{>{\raggedright\let\newline\\\arraybackslash\hspace{0pt}}m{#1}}
\newcolumntype{C}[1]{>{\centering\let\newline\\\arraybackslash\hspace{0pt}}m{#1}}
\newcolumntype{R}[1]{>{\raggedleft\let\newline\\\arraybackslash\hspace{0pt}}m{#1}}
\newcolumntype{"}{@{\hskip\tabcolsep\color{linecolor}{\vrule width 1pt}\hskip\tabcolsep}}
\makeatother

\usepackage{color}

% Link setup
\usepackage{hyperref}
\definecolor{linkcolor}{rgb}{0.2, 0.3, 0.5}
\definecolor{linecolor}{rgb}{0.5, 0.5, 0.5}
\hypersetup{unicode=true,
            pdftitle={Matthew C. Hancock},
            pdfborder={0 0 0},
            breaklinks=true,
            colorlinks,
            urlcolor=linkcolor,
            linkcolor=linkcolor}
\urlstyle{same} % don't use monospace font for urls

% No section numbers
\setcounter{secnumdepth}{0} 
% No page numbers
\pagestyle{empty}

\usepackage{setspace}

% Set font.
\usepackage{fontspec}
%\setmainfont{URW Bookman L}
\setmainfont{URW Palladio L}

\usepackage{titlesec}
\titleformat{\section}{\Large\bfseries\raggedright}{}{0em}{}[\color{linecolor}{\titlerule[1pt]}]
\titlespacing{\section}{0pt}{10pt}{20pt}

\begin{document}

\vspace*{2in}

\noindent Dear Hiring Manager,

\vspace{0.25in}

\begin{onehalfspace}
\noindent I am writing with regards to the position at BNY Mellon in the model validation group. I would appreciate the opportunity to be considered for this position. I was notified of this opening by Dr. Shangzuo Gao, a former student of my PhD advisor, Dr. Jerry Magnan.

I am presently a PhD Candidate in Applied and Computational Mathematics at Florida State University, where my research focuses on image-processing and image-analysis techniques for automated characterization of tumors in lung CT scans. Although this research might, at first, appear disconnected from the financial data analysis performed at BNY, I believe that my particular background and experience offers a unique perspective, which lends itself more generally to areas beyond my current research in medical image analysis, e.g., to data analysis in finance.

Please find my resume listed in the following pages. Should you like to discuss anything further, please feel free to contact me by phone (810.656.0561) or by email (\href{mailto:mhancock@math.fsu.edu}{mhancock@math.fsu.edu}).
\end{onehalfspace}

\vspace{0.25in}

\noindent Best,

\noindent Matthew C. Hancock

\clearpage

\section{Matthew C. Hancock}

\begin{tabular}{rl}
Address: & 317 Mabry St. Apt 1021 
           Tallahassee, FL 32304 \\
Phone: & 810.656.0561 \\
Email: & \href{mailto:mhancock@math.fsu.edu}{\nolinkurl{mhancock@math.fsu.edu}} \\
Website: & \url{https://notmatthancock.github.io}\end{tabular}

\section{Education}

\begin{tabular}{R{1in}"L{4in}}
    \emph{Fall~2012-Fall~2017\newline
(anticipated)} & \textbf{Ph.D. Candidate} in Applied and Computational Mathematics \\
    & \textbf{Florida State University} (Tallahasse, FL)
\\[4pt] & \textbf{Focus}:  Machine learning and image-processing methods for lung image analysis\\[4pt] & \textbf{Relevant coursework}: {\small Numerical Methods (interpolation, integration, ODEs/PDEs, linear algebra, and optimization), Machine Learning, Probability theory and Statistical Inference}\end{tabular}

\vspace{.15in}
\begin{tabular}{R{1in}"L{4in}}
    \emph{Spring 2012} & \textbf{B.S.} in Applied Mathematics with Computer Science focus \\
    & \textbf{Ferris State University} (Big Rapids, MI)
\end{tabular}

\vspace{.15in}


\section{Work}

\begin{tabular}{R{1.5in}"L{4in}}
    \emph{Fall 2012-present} & \textbf{Teaching Assistant} at Florida State University (Tallahassee, FL) \\
\\[4pt] & \textbf{Responsibilities / Accomplishments}:
        \begin{itemize}
        \setlength\itemsep{0pt}
        {\small
            \item Instructor, Multi-variable Calculus (Summer 2017)
            \item Distinguished Teaching Assistant Award (2017)
            \item Assistant, Foundations of Computational Math (graduate level course) (Fall 2016, Spring 2017).
            \item Instructor, C++ computing seminar (Fall 2016).
            \item Instructor, Single-variable Calculus (Spring 2016, Summer 2016).
            \item Instructor, Precalculus (Fall 2014, Spring 2015).
            \item Recitation instructor, Discrete Mathematics (Fall 2015).
            \item Assistant, various math courses (College algebra, Liberal Arts math, Trigonometry, Business calculus).
        }
        \end{itemize}
\end{tabular}

\vspace{.15in}
\begin{tabular}{R{1.5in}"L{4in}}
    \emph{Fall 2011-Fall 2012} & \textbf{Web developer} at Occupational Research and Assessment (Big Rapids, MI) \\
\\[4pt] & \textbf{Responsibilities / Accomplishments}:
        \begin{itemize}
        \setlength\itemsep{0pt}
        {\small
            \item Created and designed web systems for a number of third-party organizations using the Ruby on Rails web development framework.
        }
        \end{itemize}
\end{tabular}

\vspace{.15in}
\begin{tabular}{R{1.5in}"L{4in}}
    \emph{Fall 2009-Fall 2011} & \textbf{Programming Tutor} at Ferris State University (Big Rapids, MI) \\
\\[4pt] & \textbf{Responsibilities / Accomplishments}:
        \begin{itemize}
        \setlength\itemsep{0pt}
        {\small
            \item Tutor for undergraduate introductory programming course taught with the Python programming language.
        }
        \end{itemize}
\end{tabular}

\vspace{.15in}
\begin{tabular}{R{1.5in}"L{4in}}
    \emph{Spring 2010-Fall 2010} & \textbf{Calculus Tutor} at Ferris State University (Big Rapids, MI) \\
\\[4pt] & \textbf{Responsibilities / Accomplishments}:
        \begin{itemize}
        \setlength\itemsep{0pt}
        {\small
            \item Tutor for undergraduate calculus courses (mostly single-variable calculus material).
        }
        \end{itemize}
\end{tabular}

\vspace{.15in}


\section{Computational Fluency}

{\begin{itemize} \item High-level programming languages: Python (NumPy, SciPy, Scikit Learn, Scikit Image, Theano, Cython), Matlab/Octave, JavaScript, Ruby, PHP\end{itemize}}{\begin{itemize} \item Mid-level programming languages: C++, Fortran\end{itemize}}{\begin{itemize} \item Markup languages: \LaTeX, Web (HTML, CSS)\end{itemize}}{\begin{itemize} \item Relational databases languages: SQLite, MySQL\end{itemize}}{\begin{itemize} \item General Unix-like operating system tools\end{itemize}}

\section{Activities}

{\begin{itemize} \item {\bf Participant in Capital One modeling competition} -- Our group created a neural network model for identifying fraudulent credit card transactions. This involved preprocessing tens of gigabytes of raw data to be placed into an SQL database, whereafter a custom, GPU-based neural network was trained.\end{itemize}}{\begin{itemize} \item {\bf Creator and developer of Pylidc} -- Pylidc is a software library for working with LIDC lung CT dataset. The library is built with Python and its associated scientific computing libraries and is freely available. \url{https://github.com/pylidc/pylidc}\end{itemize}}{\begin{itemize} \item {\bf C++ and Fortran Reference Guides for Graduate Seminar} -- In collaboration with a fellow graduate student, Emacs Org-mode was used to create C++ and Fortran reference guides to be used in the Applied and Computational Mathematics computational seminar for first year graduate students at FSU. The guides are available under a Creative Commons license. \url{http://notmatthancock.github.io/teaching/acm-computing-seminar/resources/langs/cpp/}\end{itemize}}

\section{Journal Publications}

\begin{itemize}
    \item Matthew C. Hancock, Jerry F. Magnan. \textbf{Lung nodule malignancy classification using only radiologist quantified image features as inputs to statistical learning algorithms: probing the Lung Image Database Consortium dataset with two statistical learning methods}. \textit{SPIE Journal of Medical Imaging}. Dec. 2016. \url{http://dx.doi.org/10.1117/1.JMI.3.4.044504}\end{itemize}


\section{Conference Proceedings}

\begin{itemize}
    \item Matthew C. Hancock, Jerry F. Magnan. \textbf{Predictive capabilities of statistical learning methods for lung nodule malignancy classification using diagnostic image features: an investigation using the Lung Image Database Consortium dataset}. \textit{SPIE Medical Imaging Symposium, Computer-Aided Diagnosis Conference (Orlando, FL)}. Feb. 2017. \url{http://dx.doi.org/10.1117/12.2254446}\end{itemize}


\section{Talks Given}

\begin{itemize}
    \item Matthew C. Hancock, Jerry F. Magnan. \textbf{Predictive capabilities of statistical learning methods for lung nodule malignancy classification using diagnostic image features: an investigation using the Lung Image Database Consortium dataset}. \textit{SPIE Medical Imaging Symposium, Computer-Aided Diagnosis Conference (Orlando, FL)}. Feb. 2017. (talk associated with corresponding conference proceeding). \url{http://notmatthancock.github.io/research/talks/spieconf2017.pdf}\end{itemize}
\begin{itemize}
    \item Matthew C. Hancock. \textbf{10 FREE Python Libraries that will TOTALLY SHOCK you}. \textit{FSU Math Department Graduate Student Seminar}. Spring 2017. (Presentation of various Python library for scientific computing. The title is a spoof on clickbait journalism.). \url{http://notmatthancock.github.io/research/talks/gss-python/}\end{itemize}
\begin{itemize}
    \item Matthew C. Hancock. \textbf{A survey a of PDE-based methods for image segmentation }. \textit{FSU Math Department Graduate Student Seminar}. Spring 2016. \url{http://notmatthancock.github.io/research/talks/gss-pdes/}\end{itemize}


\section{Posters Presented}

\begin{itemize}
    \item Matthew C. Hancock, Jerry F. Magnan. \textbf{Lung nodule malignancy classification using diagnostic image features}. \textit{SIAM SEAS Conference (Tallahassee, FL)}. Spring 2017. \url{http://notmatthancock.github.io/research/pdf/siam-seas-2017.pdf}\end{itemize}

\vspace{0.35in}

\noindent{\bf References available upon request.} 

\end{document}
